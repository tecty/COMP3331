\documentclass{article}
\usepackage{graphicx}
\usepackage{amsmath,amsthm} 
\usepackage{amsfonts} 
\usepackage{pdfpages}
\usepackage{listings}
\lstset{breaklines} 

\begin{document}

\title{COMP3331 Lab7}
\author{Ruofei HUANG}

\maketitle

\section{Exercise 1}

\subsection{Question 1}
Subnet Table:\\
\begin{tabular}{l l l}
    Subnet&Number&Netmask\\
    \hline
    Subnet 1& 10.1.1.0 & 255.255.255.0\\
    Subnet 2& 10.1.2.0 & 255.255.255.0\\
    Subnet 3& 10.1.0.0 & 255.255.0.0\\
\end{tabular}\\
Interface Table:\\
\begin{tabular}{l r}
    Interface & Ip Address\\
    \hline
    H1    & 10.1.1.2 \\
    H2    & 10.1.1.3 \\
    H3    & 10.1.2.2 \\
    H4    & 10.1.2.3 \\
    R1a   & 10.1.1.1 \\
    R1b   & 10.1.0.2 \\
    R1c   & 10.1.2.1 \\
    NAT-i & 10.1.0.1 \\
\end{tabular}

\subsection{Question 2}

Since the IPv6 will use 128-bit of address, it can count more than all the sand on earth. 
Hence, it's impossible to run out of the address space in forseeable future. The NAT machnism
is dealing with the address is not enough in IPv4. So if every device can be directly addressed
by IPv6 in public network, there's no need to setup an NAT device in subnet.

\subsection{Question 3}

It's tedious to remember the IPv6 address and difficult to type into address bar without 
any error.

\subsection{Question 4}

HTTP as an example: \\
There's some IP address and port number inside UDP segment. If NAT leave it as what it is,
orthers may not find the correct address because it's belong to subnet and need to translate.
In this case, if NAT doesn't do the support, all the application base on this protocal wont
work.

\section{Exercise 2}

\subsection{Question 1}

192.168.1.100

\subsection{Question 2}

Source 192.168.1.100:4335\\
Destination  64.233.169.104:80

\subsection{Question 3}

At 7.158797

\subsection{Question 4}

At 7.075657

\subsection{Question 5}

Source: 54.233.169.104:80
Destination: 192.168.1100:4335
At 7.108986 


\end{document}