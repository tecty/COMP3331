\documentclass{article}
\usepackage{graphicx}
\usepackage{amsmath,amsthm} 
\usepackage{amsfonts} 
\usepackage{pdfpages}
\usepackage{listings}
\lstset{breaklines} 

\begin{document}

\title{COMP3331 Lab7}
\author{Ruofei HUANG}

\maketitle

\section{Exercise 1}

\subsection{Question 1}
Subnet Table:\\
\begin{tabular}{l l l}
    Subnet&Number&Netmask\\
    \hline
    Subnet 1& 10.1.1.0 & 255.255.255.0\\
    Subnet 2& 10.1.2.0 & 255.255.255.0\\
    Subnet 3& 10.1.0.0 & 255.255.0.0\\
\end{tabular}\\
Interface Table:\\
\begin{tabular}{l r}
    Interface & Ip Address\\
    \hline
    H1    & 10.1.1.2 \\
    H2    & 10.1.1.3 \\
    H3    & 10.1.2.2 \\
    H4    & 10.1.2.3 \\
    R1a   & 10.1.1.1 \\
    R1b   & 10.1.0.2 \\
    R1c   & 10.1.2.1 \\
    NAT-i & 10.1.0.1 \\
\end{tabular}

\subsection{Question 2}

Since the IPv6 will use 128-bit of address, it can count more than all the sand on earth. 
Hence, it's impossible to run out of the address space in forseeable future. The NAT machnism
is dealing with the address is not enough in IPv4. So if every device can be directly addressed
by IPv6 in public network, there's no need to setup an NAT device in subnet.

\subsection{Question 3}

It's tedious to remember the IPv6 address and difficult to type into address bar without 
any error.

\subsection{Question 4}

HTTP as an example: \\
There's some IP address and port number inside UDP segment. If NAT leave it as what it is,
orthers may not find the correct address because it's belong to subnet and need to translate.
In this case, if NAT doesn't do the support, all the application base on this protocal wont
work.

\section{Exercise 2}

\subsection{Question 1}

192.168.1.100

\subsection{Question 2}

Source 192.168.1.100:4335\\
Destination  64.233.169.104:80

\subsection{Question 3}

At 7.158797

\subsection{Question 4}

At 7.108986\\
Source: 192.168.1100:4335\\
Destination: 64.233.169.104:80\\

\subsection{Question 5}

Source: 64.233.169.104:80\\
Destination: 192.168.1100:4335\\
At 7.108986 

\subsection{Question 6}

At 6.069168

\subsection{Question 7}

Source 71.192.34.104:4335\\
Destination: 64.233.169.104:80\\
The destination ip and port are same as Question 2.

\subsection{Question 8}

The reponse in frame and next request in frame are changed

\subsection{Question 9}

The checksum is changed, because the checksum is include the source ip and destination, 
so it will be changed (becasue source ip is changed).

\subsection{Question 10}

At 6.117570

\subsection{Question 11}

Source: 64.233.169.104:80\\
Destination: 71.192.34.104:4335\\
The destination port and ip is different.

\subsection{Question 12}

TCP SYN at 6.035475 \\
The server to client TCP SYN/ACK at 6.067775

\subsection{Question 13}

\begin{tabular} {l l l}
    Segment Name& Source IP      & Destination IP\\
    TCP SYN     & 71.192.34.104  & 64.233.169.104\\
    TCP SYN/ACK & 64.233.169.104 & 71.192.34.104
\end{tabular}
The source of TCP SYN and the destination of TCP SYN/ACK are different.
The destination of TCP SYN and Source of TCP SYN/ACK are same.

\subsection{Question 14}
\begin{tabular} {l l}
    Source            & Destination\\
    192.168.1100:4335 & 71.192.34.104:4335
\end{tabular}\\
Maybe there's a line for:\\
\begin{tabular}{l l}
    Source            & Destination\\
    71.192.34.104:4335 & 192.168.1100:4335 
\end{tabular}

\subsection{Question 15}

Browser will check an online backlist by url. So if the URL is in the blacklist, the browser block the request.

\section{Exercise 3}

\subsection{Question 1}
Source: 00:06:25:da:af:73

\subsection{Question 2}
Destination: 00:d0:59:a9:3d:68
No, it's not, this MAC address is belong to the switch in this subnet.

\subsection{Question 3}
0x00000800

\subsection{Question 4}
0x37= 3*16+ 7 = 55,
The "G" is number 55 bytes of the Ethernet frame. So, it's 55 bytes away from the very start of the ethernet frame. \\
No preamble bytes. 14 bytes. \\
There are 41 bytes remains.

\subsection{Question 5}
The source is 00:06:25:da:af:73. Both answer are no. The address is belong to the switch of this subnet.

\section{Exercise 4}

\subsection{Question 1}

\begin{tabular}{l l l}
    No & Source            & Destination       \\
    1  & 00:d0:59:a9:3d:68 & ff:ff:ff:ff:ff:ff \\
    2  & 00:06:25:da:af:73 & 00:d0:59:a9:3d:68 \\
\end{tabular}

The address of ff:ff:ff:ff:ff:ff means broadcast, not the actural address.

\subsection{Question 2}

0x00000806

\subsection{Question 3}

(48+48+16+16+16+8+8)/8 = 20 bytes

\subsection{Question 4}

0x002

\subsection{Question 5}

Yes

\subsection{Question 6}

It's in the target IP. it's from 0x26 to 0x2A bytes, which is 38 to 42 bytes in IPv4

\subsection{Question 7}

Same as question 3? 20 bytes.

\subsection{Question 10}
The heximal contain the source and destination is:
0000   00 01 08 00 06 04 00 02 00 06 25 da af 73 c0 a8
0010   01 01 00 d0 59 a9 3d 68 c0 a8 01 69
The hex of sourcce and destination are:
\begin{tabular}{l l}
    Source & 00:06:25:da:af:73 \\
    Destination & 00:d0:59:a9:3d:68 \\
\end{tabular}


\end{document}W