\documentclass{article}
\usepackage{graphicx}
\usepackage{amsmath,amsthm} 
\usepackage{amsfonts} 
\usepackage{pdfpages}
\usepackage{listings}
\lstset{breaklines} 

\begin{document}

\title{COMP3331 Lab6}
\author{Ruofei HUANG}

\maketitle

\section{Exercise 1}
\subsection{Question 1}

Node 0 and Node 2. \\
Node 0 communicate with 5 via 1,4.\\
Node 2 communicate with 5 via 3.\\ \\
Node 0 use: 0 - 1 - 4 - 5\\
Node 2 use: 2 - 3 - 5 \\
No, the route doesn't change over time.

\subsection{Quesiton 2}

The link 1 - 4 is down. No route between the communicating nodes changed. Hence some package from node 0 has lost.

\subsection{Question 3}

There's some additional traffic which is the router tell it's neighbour about it's distance vector. \\
The router 1 re-route the traffic from node 0 when the link 1 - 4 is down to node 2. 

\subsection{Question 4}

It affect the traffic from node 0 instead of taking link 1 - 4 by taking link 1 - 2. Because this change increase the cost between 1 - 4 from 1 to 3. This will make the cost of 1 - 4 - 5 become 4 while 1 - 2 - 3 - 5 is only 3. So, it's cheaper to take the route 1 - 2 - 3 - 5 by the distance vector algorithm, even the other route has less link to pass.

\subsection{Question 5}

Result: The traffic from node 2 take two route : 2 - 1 - 4 - 5 and 2 - 3 - 5 \\
Reason: The setting of multipath is true, and the node will use mutiple paths to any distination. After this setting, both 2 - 1 - 4 - 5 and 2 - 3 - 5 has same cost 4, so the node 2 chose two path to send it's traffic to node 5.

\section{Exercise 2}

\subsection{Question 1}

Because n3 has bigger bandwitdth in n2 (10Mbps) than n0 has in n2 (2.5Mbps). It's more less likely that n2 will drop the package from n3 since it has more package arrived in n2 is from n3. Hence, after adjustment between n3 and n5 (from 2s to 6s), the throughput of tcp2 is larger than tcp1. \\
Also, there's less latency from n3 to n5 than n0 to n5, which lead to less fluctuation. And lead to more stable connection. 

\subsection{Question 2}

Because it takes time to receive the ack from 5 to 0. After sending a small amount of data (by the restriction of window size of tcp), there's no traffic in this link. 

\subsection{Question 3}

After node 0 or node 3 want to send more package, the node 2 will drop their package, which lead to their window size drop to 1 and start 'slow start'. There's a chance of TCP 1 to achieve a higher throughput, but the traffic of TCP2 comes in, and increase the package loss in TCP1 and leads to lower throughput in TCP 1. 

\section{Exercise 3}

\subsection{Question 1}

2000 and 3500 because in default, each IP segment only can contain 1480 bytes of data, any package is larger than 1480 will be fragmented to small IP segment. \\
By host 192.168.1.103.\\
Two fragments.

\subsection{Question 2}

Yes it's fragmented. Because the reply of ping is echoing the input which means we will receive a 3500 datagram and need to be fragmented into small IP segment.

\subsection{Question 3}

\begin{tabular}{ l l l l }
    ID & Length & Flag & offset \\
    \hline
    7a7b & 1514 & 0x2000, More fragments & 0 \\
    7a7b & 1514 & 0x20b9, More fragments & 185 \\
    7a7b & 582  & 0x0172 & 370\\
\end{tabular}

\section{Question 4}
Yes. 3500 > 1480 which means the MTU is smaller than the data size.

\section{Question 5}

The retransmission will occour in TCP layer which means the whole TCP package will be retransmitted. So it will retransmit 3 IP fragments.

\end{document}