\documentclass{article}
\usepackage{graphicx}
\usepackage{amsmath,amsthm} 
\usepackage{amsfonts} 
\usepackage{pdfpages}
\usepackage{listings}
\lstset{breaklines} 

\begin{document}

\title{COMP3331 Lab2}
\author{Ruofei HUANG}

\maketitle
\section{Exercise 1}
Not include.
\section{Exercise 2}

\subsection{Question 1}

Yes, google setted cookies to browsers.\\
by \\

\begin{lstlisting}
Set-Cookie: 1P_JAR=2019-03-03-07; expires=Tue, 02-Apr-2019 07:59:11 GMT; path=/; domain=.google.com
Set-Cookie: NID=162=r6tjRe1OxOkj4KuTXuh8Dag9BwMzS1Ezh1bua7fpX4DRqgnwugvboycJ-iuVJblG_O5Kl1LDjfEIRZZgaYibJD2pSDEAOJln_ri0UmhJ-zpddpAqeOaTmcFKAATHnoaI7MlwVXtOO0O5-My4T_7p6cA-gAvlYT3encdet6iAAms; expires=Mon, 02-Sep-2019 07:59:11 GMT; path=/; domain=.google.com; HttpOnly
\end{lstlisting}
Not, there's no set cookies in ucla.edu.au. But I think it will set cookies, because cookies is part of ancient web technology (now is replaced with localStorage and web browser's database such as indexedDB)

\subsection{Question 2}
There's 7 cookies setted by google.

\begin{tabular}{|r|l|}
    URL & Cookie\\
    accounts.google.com& 1\\
    google.com & 4\\
    google.com.au & 1 \\
    doubleclick.net & 1\\
\end{tabular}

\subsection{Question 3}
None. None site set cookies, Yes, it's consistence.

\end{document}


